\documentclass[a4paper,12pt]{article}

\usepackage[T1]{fontenc}
\usepackage{geometry}
\usepackage{hyperref}
\usepackage{blindtext}
\usepackage{underscore}
\usepackage{polski}
\usepackage{graphicx}
\usepackage{amssymb}
\usepackage{amsmath}
\usepackage{caption}
\usepackage{float}
\usepackage{xcolor}
\usepackage{listings}

% ----------------------------------------------------------------------------

\title{Dokumentacja projektu \\ \large Klasyfikacja nowotworu na podstawie skanu mózgu}
\author{Łukasz Jarosz}
\date{Semestr zimowy 2022/2023}

% ----------------------------------------------------------------------------

\geometry
{
  a4paper,
  total = {170mm, 257mm},
  left = 20mm,
  top = 20mm,
}

\graphicspath{ {./images/} }

% ----------------------------------------------------------------------------

\begin{document}
\begin{sloppypar}
\maketitle

\newpage
\tableofcontents
\newpage
\pagenumbering{arabic}

% ----------------------------------------------------------------------------

\section{Opis projektu}

\subsection{Cel projektu}
\paragraph{Celem projektu było napisane aplikacji, która na podstawie zdjęcia skanu mózgu jest w stanie określić, czy na danym skanie znajduje się nowotwór czy nie.}

\subsection{Użyte technologie}
\paragraph{Aplikacja została napisana w języku Python korzystając z biblioteki \href{https://www.pysimplegui.org/en/latest/}{PySimpleGui} w celu wykonania interfejsu graficznego.}


\section{Podział pracy}

\paragraph{Dokumentacja - Łukasz Jarosz}
\paragraph{Interfejs graficzny - Łukasz Jarosz}
\paragraph{Projekt sieci neuronowej - Łukasz Jarosz}
\paragraph{Trening sieci neuronowej - Łukasz Jarosz + GoogleColab}

\end{sloppypar}
\end{document}